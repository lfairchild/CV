%% start of file `template.tex'.
%% Copyright 2006-2013 Xavier Danaux (xdanaux@gmail.com).
%
% This work may be distributed and/or modified under the
% conditions of the LaTeX Project Public License version 1.3c,
% available at http://www.latex-project.org/lppl/.


\documentclass[11pt,letterpaper,sans]{moderncv}
% possible options include font size ('10pt', '11pt' and '12pt'), paper size
% ('a4paper', 'letterpaper', 'a5paper', 'legalpaper', 'executivepaper' and
% 'landscape') and font family ('sans' and 'roman')

% modern themes
\moderncvstyle{banking} % style options are 'casual' (default), 'classic', 'oldstyle' and 'banking'
\moderncvcolor{blue} % color options 'blue' (default), 'orange', 'green', 'red', 'purple', 'grey' and 'black'
%\renewcommand{\familydefault}{\rmdefault}         % to set the default font; use '\sfdefault' for the default sans serif font, '\rmdefault' for the default roman one, or any tex font name
%\nopagenumbers{}                                  % uncomment to suppress automatic page numbering for CVs longer than one page

% character encoding
\usepackage[utf8]{inputenc}                       % if you are not using xelatex ou lualatex, replace by the encoding you are using
%\usepackage{CJKutf8}                              % if you need to use CJK to typeset your resume in Chinese, Japanese or Korean

% adjust the page margins
\usepackage[scale=0.8]{geometry}
%\setlength{\hintscolumnwidth}{3cm}                % if you want to change the width of the column with the dates
% \setlength{\makecvtitlenamewidth}{30cm}           % for the 'classic' style, if you want to force the width allocated to your name and avoid line breaks. be careful though, the length is normally calculated to avoid any overlap with your personal info; use this at your own typographical risks...

\usepackage{import}
\usepackage{etaremune}
\usepackage{hyperref}
\usepackage{enumitem}
\usepackage[demo]{graphicx}
\usepackage{adjustbox}
\usepackage{multicol}
\usepackage{fontawesome}
\usepackage{xspace}
\usepackage{anyfontsize}

% align indentation of wrapped lines
\def\D{\par\noindent\makebox[1em][l]{-- }\hangindent1em}

% personal data
\name{Luke}{Fairchild}
% \title{{\fontsize{20}{24}\selectfont Curriculum Vitae}}                               % optional, remove / comment the line if not wanted
\title{Curriculum Vitae}                               % optional, remove / comment the line if not wanted
% \address{University of California, Berkeley | Department of Earth \& Planetary Science}{307 McCone Hall, Berkeley, CA 94720-4767}{}% optional, remove / comment the line if not wanted; the "postcode city" and and "country" arguments can be omitted or provided empty
% \phone[mobile]{(319) 325-5048}                   % optional, remove / comment the line if not wanted
%\phone[fixed]{01234 123456}                    % optional, remove / comment the line if not wanted
%\phone[fax]{+3~(456)~789~012}                      % optional, remove / comment the line if not wanted
\email{lukemfairchild@gmail.com}                               % optional, remove / comment the line if not wanted
\homepage{lfairchild.github.io}                         % optional, remove / comment the line if not wanted
\extrainfo{%
\raisebox{-1pt}{\large\faGithub{}}
\href{https://github.com/lfairchild}{\textsl{\small lfairchild}}
\quad
\raisebox{0pt}{\faLinkedin{}}
\href{https://www.linkedin.com/in/lukefairchild}{\textsl{\small lukefairchild}}
}
% optional, remove / comment the line if not wanted
%\photo[64pt][0.4pt]{picture}                       % optional, remove / comment the line if not wanted; '64pt' is the height the picture must be resized to, 0.4pt is the thickness of the frame around it (put it to 0pt for no frame) and 'picture' is the name of the picture file
%\quote{Some quote}                                 % optional, remove / comment the line if not wanted

% to show numerical labels in the bibliography (default is to show no labels); only useful if you make citations in your resume
%\makeatletter
%\renewcommand*{\bibliographyitemlabel}{\@biblabel{\arabic{enumiv}}}
%\makeatother
%\renewcommand*{\bibliographyitemlabel}{[\arabic{enumiv}]}% CONSIDER REPLACING THE ABOVE BY THIS

% bibliography with mutiple entries
%\usepackage{multibib}
%\newcites{book,misc}{{Books},{Others}}
\begin{document}

\makecvtitle

%\small{I am a PhD student in Earth and Planetary Science. Most generally, my
%research is focused on using the rock record to resolve the evolution of our
%planet, the continents, and the geomagnetic field over time scales of tens to
%hundreds of millions of years. I am particularly interested in the Meso- and
%Neoproterozoic Eras of Earth history (from 1600 to $\sim$540 million years
%ago). My work has sought to integrate paleomagnetism, geochronology,
%stratigraphy, and numerical models
%
%important events and geodynamic cycles that have set the trajectory of Earth
%history and large-scale changes in the configuration of continents,  Earth
%history through the integration of paleomagnetism, geochronology. Most research
%projects conducted out of the Swanson-Hysell Group
%(\color{blue}{\href{http://www.swanson-hysell.org}{swanson-hysell.org}}).}
\vspace{-20pt}
\section{Education}

\begin{itemize}

 \item{\cventry{2020--present}{Master of Science}{University of Colorado, Boulder}{Boulder,
             CO}{Electrical Engineering}{%
             Areas of study: Power electronics, Li-ion battery technology,
     battery management systems}}

 \item{\cventry{2015--2018}{Master of Science}{University of California, Berkeley}{Berkeley, CA}{Earth and
        Planetary Science}{%
        \textbf{Thesis}\quad The end of Midcontinent Rift magmatism and the
        paleogeography of Laurentia}}

 \item{\cventry{2011--2015}{Bachelor of Arts}{Carleton
             College}{Northfield, MN}{Geology (awarded with distinction)}{%
        \textbf{Thesis}\quad High temperature emplacement of clastic breccia dikes
        and implications for the development and magnetization of impact craters}%
       }  % arguments 3 to 6 can be left empty

\end{itemize}

\section{Experience}

\subsection{Work}

\begin{itemize}[itemsep=-5pt]
 \item{\cventry{2020--present}{Electrician}{Fairchild Electric}{Cedarburg, WI; Muncie,
             IN}{}{
             \begin{itemize}[itemsep=-5pt,topsep=0pt]
                 \item EV charger installation
                 \item Residential service installation
                 % \item MRI service/equipment installation 
                 \item Electrical contracting for Ball State University's new
                     collegiate gaming facility
             \end{itemize}
     }}

 \item{\cventry{2019--2021}{Website and Property Management}{Waypoint Spine}{Stevens
     Point, WI}{}{
             \begin{itemize}[itemsep=-5pt,topsep=0pt]
                 \item Website maintenance, web traffic monitoring, property maintenance
                 \item GIS analysis and map creation for prospective
                     development/expansion projects
             \end{itemize}
}}

\end{itemize}

\subsection{Teaching}

\begin{itemize}[itemsep=-5pt]
 \item{\cventry{Spring 2018}{\small{EPS 115: Stratigraphy and Earth
         History}}{Teaching Assistant}{University of California,
        Berkeley}{Prof.\ Nicholas Swanson-Hysell}{}}

 \item{\cventry{Fall 2017}{EPS 50: The Planet Earth}{Graduate Student
        Instructor}{University of California, Berkeley}{Prof.\ Michael Manga}{}}

 \item{\cventry{Spring 2015}{Petrology}{Teaching Assistant}{Carleton
        College}{Prof.\ Cameron Davidson}{}}
\end{itemize}

\subsection{Field Work}

\begin{itemize}[itemsep=-5pt]
 \item{\cventry{2017}{4 weeks}{Zavkhan Basin}{Mongolia}{}{}}

 \item{\cventry{2014, 2015, 2016 \&\ 2017}{14 weeks}{Midcontinent Rift}{Upper
        Midwestern U.S.A. \&\ Ontario, Canada}{}{}}

 \item{\cventry{2013, 2014 \&\ 2015}{5 weeks}{Slate Islands Impact
        Structure}{Ontario, Canada}{}{}}

 \item{\cventry{2013}{10 weeks}{Carleton Geology Field Camp}{New
        Zealand}{}{}}

 \item{\cventry{2013}{5 weeks}{Cannon River Watershed}{Rice County,
        Minnesota}{}{}}
\end{itemize}

\subsection{Other}

\begin{itemize}[itemsep=-5pt]
 \item{\cventry{2017--2018}{Swanson-Hysell Group}{Laboratory Safety
        Coordinator}{University of California, Berkeley}{Dept.\ of Earth
        and Planetary Science}{}}
\end{itemize}

\section{Awards}

\cventry{}{Geological Society of America}{GSA Graduate Student Research Grant}{2017}{}{Paleomagnetism of the Freda Sandstone}

\cventry{}{Earthscope AGeS Program}{EarthScope Award for Geochronology Student Research}{2016}{}{Constraining rapid paleogeographic change in the Mesoproterozoic as recorded by the North American Midcontinent Rift}

\cventry{}{University of California, Graduate Division}{Chancellor's Fellowship}{2015}{}{}

\cventry{}{Carleton College}{Class of 1963 Fellowship}{2014}{}{}

\cventry{}{Carleton College}{Kolenkow-Reitz Fellowship}{2013}{}{}

\section{Publications}
\begin{etaremune}[itemsep=3pt]

\item{Sprain, C.J., Bono, R.K., Davies, C.J., Meduri, D.G., Paterson, G.A.,
      Doubrovine, P.V., Kulakov, E.V., Hawkins, L.M.A., Pesonen, L.J., Veikkolainen, T., Smirnov,
      A., Piispa, E.J., Ots, S., \textbf{Fairchild, L.M.}, Biggin, A.J., 2022, \textit{Synthesis of numerical
      models with paleomagnetic data supports dynamo state change in the Mesozoic}: Journal of
        Geophysical Research: Solid Earth (in review).}

\item{Kulakov, E.V., Sprain, C.J., Smirnov, A.V., Biggin, A.J.,  Hawkins, L.,
     Patterson, G., \textbf{Fairchild, L.M.}, Doubrovin, P.V., 2019,
     \textit{Analysis of an updated paleointensity database (QPI-PINT) for
     65--200 Ma: Implications for the long-term history of dipole moment through
 the Mesozoic}: Journal of Geophysical Research: Solid Earth,
 {\color{cyan}\href{https://doi.org/10.1029/2018JB017287}
               {doi:10.1029/2018JB017287}}.}

 % \item \textbf{Fairchild, L.M.}, Buffett, B.A., and Biggin, A.J., 2018,
 % \textit{A stochastic coupling of geomagnetic intensity and reversal
 %  frequency}. (in preparation)

 % \item{Tikoo, S.M., Swanson-Hysell, N.L., \textbf{Fairchild, L.M.}, and
 %             Gaastra, K.M., 2018, \textit{A thermal origin for the impact-induced
 %              magnetization of the Slate Islands Impact Structure}. (in preparation)}

 \item{Swanson-Hysell, N.L., \textbf{Fairchild, L.M.}, and Slotznick, S.P.,
         2019, \textit{Primary and secondary red bed magnetization constrained
         by fluvial intraclasts}: Journal of Geophysical Research: Solid Earth,
         {\color{cyan}\href{https://doi.org/10.1029/2018JB017067}
     {doi:10.1029/2018JB017067}}.}

 \item{Swanson-Hysell, N.L., Ramezani, J., \textbf{Fairchild, L.M.}, and
         Bowring, S.A., 2019, \textit{Failed rifting and fast drifting:
             Midcontinent Rift development, Laurentia's rapid motion and the
         driver of Grenvillian orogenesis}: GSA Bulletin,
         {\color{cyan}\href{https://doi.org/10.1130/B31944.1}
     {doi:10.1130/B31944.1}}.}

 \item{Sprain, C.J., Swanson-Hysell, N.L., \textbf{Fairchild, L.M.}, and
         Gaastra, K., 2018, \textit{A field like today's? The geomagnetic field
         1.1 billion years ago}: Geophysical Journal International,
         {\color{cyan}\href{https://doi.org/10.1093/gji/ggy074}
     {doi:10.1093/gji/ggy074}}.}

 \item{\textbf{Fairchild, L.M.}, Swanson-Hysell, N.L., Ramezani, J., Sprain,
             C.J., and Bowring, S.A., 2017, \textit{The end of Midcontinent Rift
              magmatism and the paleogeography of Laurentia}: Lithosphere,
             {\color{cyan}\href{https://doi.org/10.1130/L580.1}
               {doi:10.1130/L580.1}}.}

 \item{\textbf{Fairchild, L.M.}, Swanson-Hysell, N.L., and Tikoo, S.M., 2016,
             \textit{A matter of minutes: Breccia dike paleomagnetism provides
              evidence for rapid crater modification}: Geology,
             {\color{cyan}\href{https://doi.org/10.1130/G37927.1}
               {doi:10.1130/G37927.1}}.}

 \item{Bezaeva, N.S., Swanson Hysell, N.L., Tikoo, S.M., Badyukov, D.D.,
             Kars, M., Egli, R., Chareev, D.A., \textbf{Fairchild, L.M.},
             Khakhalova, E., Strauss, B.E., and Lindquist, A.K., 2016,
             \textit{The effect of 10 to >160 GPa spherically convergent shock
              waves on the magnetic properties of basalt of diabase}:
             Geochemistry, Geophysics, Geosystems,
             {\color{cyan}\href{https://doi.org/10.1002/2016GC006583}
               {doi:10.1002/2016GC006583}}.}

 \item{Tauxe, L., Shaar, R., Jonestrask, L., Swanson-Hysell, N.L., Minnett,
             R., Koppers, A.A.P., Constable, C.G., Jarboe, N., Gaastra,  K.,
             \textbf{Fairchild, L.M.}, 2016, \textit{PmagPy: Software package for
              paleomagnetic data analysis and a bridge to the Magnetics
              Information Consortium (MagIC) Database}: Geochemistry,
             Geophysics, Geosystems,
             {\color{cyan}\href{https://doi.org/10.1002/2016GC006307}
               {doi:10.1002/2016GC006307}}.}

\end{etaremune}

\section{Conference Abstracts}

\begin{itemize}
 \setlength{\itemsep}{2pt}

 \item{Swanson-Hysell, N.L., \textbf{Fairchild, L.M.}, Ramenzani, J., 2018,
       \textit{Chronostratigraphy of Midcontinent Rift volcanics provides
        new insight on rift development and the rate of rapid
        paleogeographic change}, Abstract 323067, GSA Annual Meeting.}

 \item{Kulakov, E.V., Smirnov, A.V., Biggin, A.J., Sprain, C.J., Hawkins, L.,
       Patterson, G., \textbf{Fairchild, L.M.}, 2018, \textit{The long-term
        history of the Mesozoic-Jurassic geodynamo: A paleointensity
        perspective}, European Geosciences Union General Assembly, Vienna,
       Austria.}

 \item{\textbf{Fairchild, L.M.}, Buffett, B., Biggin, A., 2017,
       \textit{Stochastic models and the absolute paleointensity (PINT)
        database: a new look at geomagnetic reversal rates}, 2017 Nordic
       Paleomagnetism Workshop, Leirubakki, Iceland.}

 \item{\textbf{Fairchild, L.M.}, Swanson-Hysell, N.L., Ramenzani, J., Sprain,
       C., Gaastra, K., Bowring, S., 2017, \textit{The end of Midcontinent
        Rift magmatism and the paleogeography of Laurentia}, 2017 Magnetics
       Information Consortium (MagIC) Workshop, La Jolla, California.}

 \item{\textbf{Fairchild, L.M.}, Swanson-Hysell, N.L., Ramenzani, J., Sprain,
       C., Gaastra, K., Bowring, S., 2016, \textit{The end of Midcontinent
        Rift magmatism and the paleogeography of Laurentia}, Abstract
       283146, GSA Annual Meeting.}

 \item{Swanson-Hysell, N.L., Ramenzani, J., \textbf{Fairchild, L.M.}, Rose,
       I., 2016, \textit{New geochronologic and paleomagnetic constraints
        on Midcontinent Rift development}, Abstract 284544, GSA Annual
       Meeting.}

 \item{Sprain, C.J., Swanson-Hysell, N.L., \textbf{Fairchild, L.M.}, Gaastra,
       K., 2016, \textit{The strength of the Mesoproterozoic geomagnetic
        field: new absolute paleointensity estimates from
        \textasciitilde 1.1 billion-year-old Midcontinent Rift
        volcanics}, Abstract 154089, AGU Fall Meeting.}

 \item{Bezaeva, N.S., Swanson-Hysell, N.L., Tikoo, S.M., Kars, M., Egli, R.,
       Badyukov, D.D., Chareev, D.A., Fairchild L.M., 2016,
       \textit{Discrimination of Thermal versus Mechanical Effects of Shock
        on Rock Magnetic Properties of Spherically Shocked up to
        \textasciitilde 10--160 GPa Basalt and Diabase}, Abstract
       GP31A-1282, AGU Fall Meeting.}

 \item{Bezaeva, N.S., Swanson-Hysell, N.L., Tikoo, S.M., Kars, M., Egli, R.,
       Badyukov, D.D., Chareev, D.A., \textbf{Fairchild, L.M.}, 2016,
       \textit{How to discriminate between thermal and mechanical effects
        of shock on the rock magnetic properties of basalt and diabase
        spherically shocked up to \textasciitilde 10--160 GPa.} Book of
       Abstracts of the 11th International Conference and School ``Problems
       of Geocosmos", October 3--7, 2016, St Petersburg, Petrodvorets,
       Russia, 126--127.}

 \item{\textbf{Fairchild, L.M.}, Swanson-Hysell, N.L., Ramenzani, J., Sprain,
       C., Gaastra, K., Bowring, S., 2015, \textit{When did Midcontinent
        Rift volcanism end and where was Laurentia at that time?} Abstract
       GP31A-1364, AGU Fall Meeting.}

 \item{Bezaeva, N.S., Swanson-Hysell, N.L., Tikoo, S.M., Badyukov, D., Kars,
       M., Egli, R., Chareev, D., \textbf{Fairchild, L.M.}, Khakhalova, E.,
       Strauss, B., and Lindquist, A., 2015, \textit{Rock magnetic effects
        induced in terrestrial basalt and diabase by >20 GPa
        experimental spherical shock waves.} Abstract GP43A-1233, AGU
       Fall Meeting.}

 \item{Tikoo, S.M., Swanson-Hysell, N.L., \textbf{Fairchild, L.M.}, Renne,
       P.R., and Schuster, D.L., 2015, \textit{Origins of impact-related
        magnetization at the Slate Islands impact structure, Canada.}
       Abstract 2474, 46th Lunar and Planetary Science Conference.}

 \item{\textbf{Fairchild, L.M.}, Swanson-Hysell, N.L., Tikoo, S.M., 2014,
       \textit{High temperature emplacement of clastic breccia dikes and
        implications for the development and magnetization of impact
        craters.} Abstract 19163, AGU Fall Meeting.}

 \item{Tikoo, S.M., Swanson-Hysell, N.L., \textbf{Fairchild, L.M.}, Renne,
       P.R., and Schuster, D.L., 2014, \textit{Testing the shock remanent
        magnetization hypothesis at the Slate Islands impact structure,
        Canada.} Abstract 23778, AGU Fall Meeting.}

\end{itemize}


\section{Memberships}

\cventry{}{Supercontinent Cycles \& Global Geodynamics}{International Geoscience Programme (IGCP) 648}{2015}{}{}
\cventry{}{}{Geological Society of America}{since 2016}{}{}
\vspace{-10pt}
\cventry{}{}{American Geophysical Union}{since 2013}{}{}
\vspace{-10pt}
\cventry{}{}{Sigma Xi Research Society}{since 2015}{}{}
\vspace{-15pt}

\section{Technical and Personal Skills}
% \vspace{6pt}
% \item \textbf{Laboratory:} 2G Superconducting Quantum Interference Device
%     (SQuID) magnetometer; Kappabridge Magnetic Susceptibility/Anisotropy
%     System; X-ray diffractometer; gravimeter; scanning electron microscope;
%     petrographic analysis.
% \vspace{6pt}
\subsection{Programming Languages} \vspace{5pt}
\begin{itemize}[itemsep=0pt]
    % \item {\begin{minipage}[t]{\linewidth}
    %        \raggedright
    %        \adjustbox{valign=t}{%
    %         \includegraphics[width=.12\textwidth]{media/python-logo-vector.pdf}%
    %        }
    %       \vspace{-5pt}
    %       \end{minipage}}
    %       \raggedright
    %       \begin{itemize}
    %        \item[] Proficient with a variety of scientific computing and data analysis tools written in Python. Experience with software development, including the elopment of graphical user interface (GUI) applications with wxPython/Phoenix. Active contributor to PmagPy ({\color{cyan}\href{https://github.com/PmagPy}{github.com/PmagPy}}), an open-source software package widely used within the geomagnetism and paleomagnetism research communities that also provides the core data analysis and visualization tools underlying the Magnetics Information Consotium (MagIC) database, an online archive of paleomagnetic data ({\color{cyan}\href{https://www2.earthref.org/MagIC}{earthref.org/MagIC}}).
    %              An overview of the PmagPy project is provided in Tauxe et al.\
    %              (2016) (listed above in Publications).
    %       \end{itemize}
    \item \textbf{Python}\\
        Proficient with a variety of scientific computing, data analysis, and
        visualization tools.

    \item \textbf{Bash} and other Unix command line interface (CLI) tools

   \item \textbf{MATLAB}
   \item Limited experience with C and C\texttt{++} from miscellaneous projects (no formal training)
 \item Some experience with frontend web development (CSS, Javascript) and
       static site generators (Jekyll, GitHub Pages)
       \item Experience with SQL
\end{itemize}
% \vspace{-10pt}
% \pagebreak[4]
\subsection{Markup Languages}
\vspace{-10pt}
\begin{multicols}{3}
 \begin{itemize}%[itemsep=-3pt]
  \item {\large\fontfamily{cmr}\selectfont\LaTeX}\\
        \raggedright\normalsize
        % \subitem Extensive experience
  \item Markdown
  \item HTML
 \end{itemize}
\end{multicols}
\vspace{-15pt}

\subsection{Industry Software Skills}
%\underline{Geographic Information Systems (GIS)}
%\vspace{-5pt}
%\begin{multicols}{2}
% \begin{itemize}[itemsep=-3pt]
%  % general recommendations advise against including such logos as below...
%  %
%  % \item {\begin{minipage}[t]{\linewidth} \raggedright
%  %             \adjustbox{valign=t}{%
%  %             \includegraphics[width=.17\textwidth]{media/qgis-logo.png}}
%  %     \end{minipage}}
%  %     \raggedright Proficient and highly experienced. I have
%  %     used QGIS over many years to integrate field observations
%  %     and structural measurements in creating geologic maps, investigate
%  %     geographic trends in laboratory data, and develop spatial and temporal
%  %     models consistent with the data.\columnbreak
%  % \item {\begin{minipage}[t]{\linewidth} \raggedright
%  %             \adjustbox{valign=t}{%
%  %             \includegraphics[width=.25\textwidth]{media/arcgis_horizontal_logo.png}}
%  %     \end{minipage}} \raggedright While ultimately more familiar with the QGIS
%  %     interface, I do also have ample experience with ArcGIS in generating
%  %     custom maps and referencing high-resolution imagery. % \item ArcGIS
%  \item \textbf{QGIS}\\
%        \raggedright Proficient and highly experienced. I have
%        used QGIS over many years to integrate field observations
%        and structural measurements in creating geologic maps, investigate
%        geographic trends in laboratory data, and develop spatial and temporal
%        (paleo)geographic models.
%        \columnbreak
%  \item \textbf{ArcGIS}\\
%        \raggedright While ultimately more familiar with the QGIS
%        interface, I also have ample experience with ArcGIS in generating
%        custom maps and referencing high-resolution imagery. % \item ArcGIS
% \end{itemize}
%\end{multicols}
%\vspace{-15pt}
\begin{itemize}[itemsep=-3pt]
    \item \textbf{GIS}\\ 
        Proficient and highly experienced. I have used GIS software, including
        QGIS and ArcGIS, over many years to integrate field observations and
        structural measurements in creating geologic maps, investigate
        geographic trends in laboratory data, develop spatial and temporal
        (paleo)geographic models, georeference raster data, and many other tasks
        for which GIS software is used.
        
    \item \textbf{GPlates} (paleogeographic models, global tectonic simulation/animation)
        
    \item Adobe Creative Suite software (\textbf{Illustrator,
        Photoshop, Lightroom})
 \item \textbf{Git} version control software and online platforms for
     remote repositories and collaboration such as GitHub
   \item \textbf{LTSpice} circuit simulation
       % \item Fluent in \raisebox{-2pt}{\faGit{}} version control software and
       %     related web-based hosting services such as GitHub
\end{itemize}

\subsection{Field Skills}

\begin{itemize}
 \item Geologic mapping
 \item Rock core drilling/orienting
 \item Stratigraphic measurement and geologic structural analysis
 \item Field work logistics and planning
\end{itemize}

% \subsection{Laboratory Skills}

% \begin{itemize}
%     \item Geologic mapping
%     \item Rock core drilling/orienting
%     \item Geological structural analysis
%     \item Field work logistics and planning
% \end{itemize}

\end{document}


%% end of file `template.tex'.
