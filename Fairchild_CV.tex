%% start of file `template.tex'.
%% Copyright 2006-2013 Xavier Danaux (xdanaux@gmail.com).
%
% This work may be distributed and/or modified under the
% conditions of the LaTeX Project Public License version 1.3c,
% available at http://www.latex-project.org/lppl/.


\documentclass[11pt,a4paper,sans]{moderncv}        % possible options include font size ('10pt', '11pt' and '12pt'), paper size ('a4paper', 'letterpaper', 'a5paper', 'legalpaper', 'executivepaper' and 'landscape') and font family ('sans' and 'roman')

% modern themes
\moderncvstyle{banking}                            % style options are 'casual' (default), 'classic', 'oldstyle' and 'banking'
\moderncvcolor{blue}                                % color options 'blue' (default), 'orange', 'green', 'red', 'purple', 'grey' and 'black'
%\renewcommand{\familydefault}{\sfdefault}         % to set the default font; use '\sfdefault' for the default sans serif font, '\rmdefault' for the default roman one, or any tex font name
%\nopagenumbers{}                                  % uncomment to suppress automatic page numbering for CVs longer than one page

% character encoding
\usepackage[utf8]{inputenc}                       % if you are not using xelatex ou lualatex, replace by the encoding you are using
%\usepackage{CJKutf8}                              % if you need to use CJK to typeset your resume in Chinese, Japanese or Korean

% adjust the page margins
\usepackage[scale=0.75]{geometry}
%\setlength{\hintscolumnwidth}{3cm}                % if you want to change the width of the column with the dates
%\setlength{\makecvtitlenamewidth}{10cm}           % for the 'classic' style, if you want to force the width allocated to your name and avoid line breaks. be careful though, the length is normally calculated to avoid any overlap with your personal info; use this at your own typographical risks...

\usepackage{import}
\usepackage{etaremune}

% personal data
\name{Luke}{Fairchild}
\title{Curriculum Vitae}                               % optional, remove / comment the line if not wanted
\address{University of California, Berkeley | Department of Earth \& Planetary Science}{307 McCone Hall, Berkeley, CA 94720-4767}{}% optional, remove / comment the line if not wanted; the "postcode city" and and "country" arguments can be omitted or provided empty
\phone[mobile]{(319) 325-5048}                   % optional, remove / comment the line if not wanted
%\phone[fixed]{01234 123456}                    % optional, remove / comment the line if not wanted
%\phone[fax]{+3~(456)~789~012}                      % optional, remove / comment the line if not wanted
\email{lfairchild@berkeley.edu}                               % optional, remove / comment the line if not wanted
%\homepage{www.myname.webs.com}                         % optional, remove / comment the line if not wanted
%\extrainfo{additional information}                 % optional, remove / comment the line if not wanted
%\photo[64pt][0.4pt]{picture}                       % optional, remove / comment the line if not wanted; '64pt' is the height the picture must be resized to, 0.4pt is the thickness of the frame around it (put it to 0pt for no frame) and 'picture' is the name of the picture file
%\quote{Some quote}                                 % optional, remove / comment the line if not wanted

% to show numerical labels in the bibliography (default is to show no labels); only useful if you make citations in your resume
%\makeatletter
%\renewcommand*{\bibliographyitemlabel}{\@biblabel{\arabic{enumiv}}}
%\makeatother
%\renewcommand*{\bibliographyitemlabel}{[\arabic{enumiv}]}% CONSIDER REPLACING THE ABOVE BY THIS

% bibliography with mutiple entries
%\usepackage{multibib}
%\newcites{book,misc}{{Books},{Others}}
%----------------------------------------------------------------------------------
%            content
%----------------------------------------------------------------------------------
\begin{document}
%\begin{CJK*}{UTF8}{gbsn}                          % to typeset your resume in Chinese using CJK
%-----       resume       ---------------------------------------------------------
\makecvtitle

%\small{I am a PhD student in Earth and Planetary Science. Most generally, my research is focused on using the rock record to resolve the evolution of our planet, the continents, and the geomagnetic field over time scales of tens to hundreds of millions of years. I am particularly interested in the Meso- and Neoproterozoic Eras of Earth history (from 1600 to $\sim$540 million years ago). My work has sought to integrate paleomagnetism, geochronology, stratigraphy, and numerical models 
%
%important events and geodynamic cycles that have set the trajectory of Earth history and large-scale changes in the configuration of continents,  Earth history through the integration of paleomagnetism, geochronology. Most research projects conducted out of the Swanson-Hysell Group (\color{blue}{\href{http://www.swanson-hysell.org}{swanson-hysell.org}}).}

\section{Education}

\vspace{5pt}

%\subsection{Academic Qualifications}

\vspace{5pt}

\begin{itemize}

\item{\cventry{2015--present}{PhD Student}{University of California, Berkeley}{Berkeley, CA}{\textit{Earth and Planetary Sciences}}{}}

\item{\cventry{2011--2015}{Undergraduate Student}{Carleton College}{Northfield, MN}{\textit{Geology (Honors), cum laude}}{Thesis: \textit{High temperature emplacement of clastic breccia dikes and implications for the development and magnetization of impact craters}}}  % arguments 3 to 6 can be left empty

%\item{\cventry{2002--2009}{11 GCSEs}{Southlands High School}{Chorley}{\textit{A* to B Including Maths and English}}{}}

\end{itemize}

\vspace{2pt}

\section{Experience}

\vspace{6pt}

\subsection{Teaching}

\vspace{5pt}

\begin{itemize}

\item{\cventry{Spring 2018}{\footnotesize{EPS 115: Stratigraphy and Earth History}}{Teaching Assistant}{University of California, Berkeley}{Prof. Nicholas Swanson-Hysell}{}}

\item{\cventry{Fall 2017}{EPS 50: The Planet Earth}{Graduate Student Instructor}{University of California, Berkeley}{Prof. Michael Manga}{}}

\item{\cventry{Spring 2015}{Petrology}{Teaching Assistant}{Carleton College}{Prof. Cameron Davidson}{}}


\end{itemize}
\vspace{6pt}


\subsection{Field Work}

\vspace{5pt}

\begin{itemize}

\item{\cventry{2017}{4 weeks}{Zavkhan Basin}{Mongolia}{}{}}

\item{\cventry{2014, 2015, 2016 \&\ 2017}{14 weeks}{Midcontinent Rift}{Upper Midwestern U.S.A. \&\ Ontario, Canada}{}{}}

\item{\cventry{2013, 2014 \&\ 2015}{5 weeks}{Slate Islands Impact Structure}{Ontario, Canada}{}{}}

\item{\cventry{2013}{10 weeks}{Carleton Geology Field Camp}{New Zealand}{}{}}

\item{\cventry{2013}{5 weeks}{Cannon River Watershed}{Rice County, Minnesota}{}{}}

\end{itemize}

\subsection{Other}

\vspace{5pt}

\begin{itemize}

\item{\cventry{2017--present}{Swanson-Hysell Group}{Laboratory Safety Coordinator}{University of California, Berkeley}{Dept. of Earth and Planetary Science}{}}

\end{itemize}

\section{Awards}

\cventry{}{Geological Society of America}{GSA Graduate Student Research Grant}{2017}{}{Paleomagnetism of the Freda Sandstone}\\

\cventry{}{Earthscope AGeS Program}{EarthScope Award for Geochronology Student Research}{2016}{}{Constraining rapid paleogeographic change in the Mesoproterozoic as recorded by the North American Midcontinent Rift}\\

\cventry{}{University of California -- Graduate Division}{Chancellor's Fellowship}{2015}{}{}\\

\cventry{}{Carleton College}{Class of 1963 Fellowship}{2014}{}{}\\

\cventry{}{Carleton College}{Kolenkow-Reitz Fellowship}{2013}{}{}\\

%\newpage

\section{Publications}

\begin{etaremune}

\vspace{6pt}

\item{\textbf{Fairchild, L.M.} and Buffett, B.A., 2018, \textit{A stochastic coupling of geomagnetic intensity and reversal frequency}: Nature Geoscience. (in preparation)}

\vspace{3pt}

\item{Tikoo, S.M., Swanson-Hysell, N.L., \textbf{Fairchild, L.M.}, and Gaastra, K.M., 2018, \textit{A thermal origin for the impact-induced magnetization of the Slate Islands Impact Structure}: Nature Geoscience. (in preparation)}

\vspace{3pt}

\item{Swanson-Hysell, N.L., Ramezani, J., \textbf{Fairchild, L.M.}, and Bowring, S.A., 2017, \textit{Failed rifting and fast drifting: Midcontinent Rift development, Laurentia's rapid motion and the driver of Grenvillian orogenesis}: GSA Bulletin. (in review)}

\vspace{3pt}

\item{Sprain, C.J., Swanson-Hysell, N.L., \textbf{Fairchild, L.M.}, and Gaastra, K., 2017, \textit{A field like today's? The geomagnetic field 1.1 billion years ago}: Geophysical Journal International.}

\vspace{3pt}

\item{\textbf{Fairchild, L.M.}, Swanson-Hysell, N.L., Ramezani, J., Sprain, C.J., and Bowring, S.A., 2017, \textit{The end of Midcontinent Rift magmatism and the paleogeography of Laurentia}: Lithosphere.}

\vspace{3pt}

\item{\textbf{Fairchild, L.M.}, Swanson-Hysell, N.L., and Tikoo, S.M., 2016, \textit{A matter of minutes: Breccia dike paleomagnetism provides evidence for rapid crater modification}: Geology.}

\vspace{3pt}

\item{Bezaeva, N.S., Swanson Hysell, N.L., Tikoo, S.M., Badyukov, D.D., Kars, M., Egli, R., Chareev, D.A., \textbf{Fairchild, L.M.}, Khakhalova, E., Strauss, B.E., and Lindquist, A.K., 2016, \textit{The effect of 10 to >160 GPa spherically convergent shock waves on the magnetic properties of basalt of diabase}: Geochemistry, Geophysics, Geosystems.}

\vspace{3pt}

\item{Tauxe, L., Shaar, R., Jonestrask, L., Swanson-Hysell, N.L., Minnett, R., Koppers, A.A.P., Constable, C.G., Jarboe, N., Gaastra,  K., \textbf{Fairchild, L.M.}, 2016, \textit{PmagPy: Software package for paleomagnetic data analysis and a bridge to the Magnetics Information Consortium (MagIC) Database}: Geochemistry, Geophysics, Geosystems.}

\end{etaremune}

\section{Conference Abstracts}

\begin{itemize}

\vspace{6pt}

\item{Kulakov, E.V., Smirnov, A.V., Biggin, A.J., Sprain, C.J., Hawkins, L., Patterson, G., \textbf{Fairchild, L.M.}, 2018, \textit{The long-term history of the Mesozoic-Jurassic geodynamo: A paleointensity perspective}, European Geosciences Union General Assembly, Vienna, Austria.}

\vspace{3pt}

\item{\textbf{Fairchild, L.M.}, Buffett, B., Biggin, A., 2017, \textit{Stochastic models and the absolute paleointensity (PINT) database: a new look at geomagnetic reversal rates}, 2017 Nordic Paleomagnetism Workshop, Leirubakki, Iceland.}

\vspace{3pt}

\item{\textbf{Fairchild, L.M.}, Swanson-Hysell, N.L., Ramenzani, J., Sprain, C., Gaastra, K., Bowring, S., 2017, \textit{The end of Midcontinent Rift magmatism and the paleogeography of Laurentia}, 2017 Magnetics Information Consortium (MagIC) Workshop, La Jolla, California.}

\vspace{3pt}

\item{\textbf{Fairchild, L.M.}, Swanson-Hysell, N.L., Ramenzani, J., Sprain, C., Gaastra, K., Bowring, S., 2016, \textit{The end of Midcontinent Rift magmatism and the paleogeography of Laurentia}, Abstract 283146, GSA Annual Meeting.}

\vspace{3pt}

\item{Swanson-Hysell, N.L., Ramenzani, J., \textbf{Fairchild, L.M.}, Rose, I., 2016, \textit{New geochronologic and paleomagnetic constraints on Midcontinent Rift development}, Abstract 284544, GSA Annual Meeting.}

\vspace{3pt}

\item{Sprain, C.J., Swanson-Hysell, N.L., \textbf{Fairchild, L.M.}, Gaastra, K., 2016, \textit{The strength of the Mesoproterozoic geomagnetic field: new absolute paleointensity estimates from $\sim$1.1 billion-year-old Midcontinent Rift volcanics}, Abstract 154089, AGU Fall Meeting.}

\vspace{3pt}

\item{Bezaeva, N.S., Swanson-Hysell, N.L., Tikoo, S.M., Kars, M., Egli, R., Badyukov, D.D., Chareev, D.A., Fairchild L.M., 2016, \textit{Discrimination of Thermal versus Mechanical Effects of Shock on Rock Magnetic Properties of Spherically Shocked up to $\sim$10--160 GPa Basalt and Diabase}, Abstract GP31A-1282, AGU Fall Meeting.}

\vspace{3pt}

\item{Bezaeva, N.S., Swanson-Hysell, N.L., Tikoo, S.M., Kars, M., Egli, R., Badyukov, D.D., Chareev, D.A., \textbf{Fairchild, L.M.}, 2016, \textit{How to discriminate between thermal and mechanical effects of shock on the rock magnetic properties of basalt and diabase spherically shocked up to $\sim$10--160 GPa.} Book of Abstracts of the 11th International Conference and School ``Problems of Geocosmos", October 3--7, 2016, St Petersburg, Petrodvorets, Russia, 126--127.}

\vspace{3pt}

\item{\textbf{Fairchild, L.M.}, Swanson-Hysell, N.L., Ramenzani, J., Sprain, C., Gaastra, K., Bowring, S., 2015, \textit{When did Midcontinent Rift volcanism end and where was Laurentia at that time?} Abstract GP31A-1364, AGU Fall Meeting.}


\vspace{3pt}

\item{Bezaeva, N.S., Swanson-Hysell, N.L., Tikoo, S.M., Badyukov, D., Kars, M., Egli, R., Chareev, D., \textbf{Fairchild, L.M.}, Khakhalova, E., Strauss, B., and Lindquist, A., 2015, \textit{Rock magnetic effects induced in terrestrial basalt and diabase by >20 GPa experimental spherical shock waves.} Abstract GP43A-1233, AGU Fall Meeting.}

\vspace{3pt}

\item{Tikoo, S.M., Swanson-Hysell, N.L., \textbf{Fairchild, L.M.}, Renne, P.R., and Schuster, D.L., 2015, \textit{Origins of impact-related magnetization at the Slate Islands impact structure, Canada.} Abstract 2474, 46th Lunar and Planetary Science Conference.}

\vspace{3pt}

\item{\textbf{Fairchild, L.M.}, Swanson-Hysell, N.L., Tikoo, S.M., 2014, \textit{High temperature emplacement of clastic breccia dikes and implications for the development and magnetization of impact craters.} Abstract 19163, AGU Fall Meeting.}

\vspace{3pt}

\item{Tikoo, S.M., Swanson-Hysell, N.L., \textbf{Fairchild, L.M.}, Renne, P.R., and Schuster, D.L., 2014, \textit{Testing the shock remanent magnetization hypothesis at the Slate Islands impact structure, Canada.} Abstract 23778, AGU Fall Meeting.}

\end{itemize}

\section{Memberships}

\cventry{}{Supercontinent Cycles \& Global Geodynamics}{International Geoscience Programme (IGCP) 648}{2015}{}{}

\cventry{}{}{Geological Society of America}{since 2016}{}{}

\cventry{}{}{American Geophysical Union}{since 2013}{}{}

\cventry{}{}{Sigma Xi Research Society}{since 2015}{}{}


\section{Technical and Personal skills}

\vspace{6pt}

\begin{itemize}

%\item \textbf{Laboratory:} 2G Superconducting Quantum Interference Device (SQuID) magnetometer; Kappabridge Magnetic Susceptibility/Anisotropy System; X-ray diffractometer; gravimeter; scanning electron microscope; petrographic analysis.
%
%\vspace{6pt}

\item \textbf{Programming Languages:} Proficient in Python, LaTeX, Jupyter notebooks, some Matlab

\vspace{6pt}

\item \textbf{Industry Software Skills:} GIS, Adobe Illustrator, Adobe Photoshop, MS Office products

\vspace{6pt}

\item \textbf{Field Skills:} Geologic mapping, rock core drilling/orienting, structural analysis

\end{itemize}

%\section{Interests and extra-curricular activity}
%
%\vspace{6pt}
%
%\begin{itemize}
%
%\item{I was a "fresher representative" in my 2nd and 3rd years of university, this required me to guide, look after, and ensure that a particular flat of first years have a good time in their first week, and feel consoled in what for most of them is there first time living away from home. We were responsible for the safety and wellbeing of the group of first years during the first week, and during this time I made good friends with all of them.}
%
%\vspace{6pt}
%
%\item{I am a member of a number of university societies. I was also the vice president and co-founder of the flash mob society. My roles in this included recruiting members, in which during "fresher's fair" we enlisted over 200 new members. This was regarded as very successful, considering other societies averaged around 50. I also appeared in an interview on the university television station, set up a society bank account, and helped organise the events. One of these events was featured in the local newspaper.}
%
%\vspace{6pt}
%
%\item{I am also an avid hiker, having completed the national 3 peaks challenge last summer. Other interest include guitar, which I am self-taught, and home brewing.}
%
%\end{itemize}
%
%\section{References}
%
%\vspace{6pt}
% 
%\begin{itemize}
%
%\item{Up to 4 references available on request}
%
%\end{itemize}

% Publications from a BibTeX file without multibib
%  for numerical labels: \renewcommand{\bibliographyitemlabel}{\@biblabel{\arabic{enumiv}}}% CONSIDER MERGING WITH PREAMBLE PART
%  to redefine the heading string ("Publications"): \renewcommand{\refname}{Articles}
\nocite{*}
\bibliographystyle{plain}
\bibliography{publications}                        % 'publications' is the name of a BibTeX file

% Publications from a BibTeX file using the multibib package
%\section{Publications}
%\nocitebook{book1,book2}
%\bibliographystylebook{plain}
%\bibliographybook{publications}                   % 'publications' is the name of a BibTeX file
%\nocitemisc{misc1,misc2,misc3}
%\bibliographystylemisc{plain}
%\bibliographymisc{publications}                   % 'publications' is the name of a BibTeX file

%-----       letter       ---------------------------------------------------------

\end{document}


%% end of file `template.tex'.
